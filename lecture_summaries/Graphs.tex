\documentclass{article}
\usepackage[utf8]{inputenc}
\usepackage{amsmath}

\title{CS120 Review: Graphs}
\author{NJD}
\date{2022}

\usepackage{listings}

\newenvironment{tightcenter}{%
  \setlength\topsep{0pt}
  \setlength\parskip{0pt}
  \begin{center}
}{%
  \end{center}
}

\begin{document}
\maketitle

\section*{Lectures 9 and 10}
\paragraph*{Definitions} |
\newline
\newline
\textbf{Directed Graph}
\begin{itemize}
  \item Defined as G = (V, E)
  \item For each $(u, v) \in E$, $u, v \in V$, $u$ "points to" $v$
\end{itemize}
\textbf{Undirected Graph} 
\begin{itemize}
  \item Defined as G = (V, E)
  \item For each $(u, v) \in E$, $u, v \in V$, $u$ and $v$ "point to each outer"
\end{itemize}
\textbf{Digraph}
\begin{itemize}
  \item Simple, unweighted, directed graph
\end{itemize}
\paragraph*{Keywords} |
\newline
\newline
\emph{Planar}: a graph can be drawn in 2D with no edge crossings.
\newline
\newline
\emph{Walk}: a sequence of verticies from $s$ to $t$
\newline
\newline
\emph{Shortest Walk}: the "distance" of $s$ to $t$ (aka, the minimum of the possible lengths)
\paragraph*{Theorems and Lemmas} |
\newline
\newline
\emph{Shortest Walk Lemma: } If $w$ is a shortest walk from $s$ to $t$, then all of the vertices that occur on w are distinct.
That is, every shortest walk is a \emph{path}.
\begin{itemize}
  \item Suppose we have a shorest walk with repeated vertices. We know there exists a shorter one
  by simply getting rid of all vertices between the first instance of the repeated vertex and the second
  for all vertices.
\end{itemize}
\emph{BFS 2-Coloring Theorems} BFSColoring on a 2-colorable graph will always return a valid 2-coloring in O(n + m) time
\begin{itemize}
  \item Think about it: if the graph is 2-colorable, each new frontier will have alternating colors
  \item Time O(n + m) is achieved through connected components 
\end{itemize}

\paragraph*{Algorithms} |
\newline
\newline
\emph{Shortest Walk}
\begin{itemize}
  \item Inputs: digraph $G = (V,E)$, verticies $s,t \in V$ 
  \item Outputs: shortest walk iff it exists
  \item Possible solving algorithms:
  \begin{enumerate}
    \item Exhaustive Search: $(n-1)! \cdot O(n)$
    \begin{itemize}
      \item Get all walks up to length n-1
      \item By Shortest Walk Lemma, our shortest walk must be here
      \item Find shortest walk starting from $s$ and ending at $t$
    \end{itemize}
    \item BFS: $O(n + m)$
    \begin{itemize}
      \item Initialize array $V$, where $V[v]$ represents which vertex we got to $v$ from
      \item From i in 0 to n-1:
      \begin{itemize}
        \item If $V[t]$, return the sequence from $V[t]$ down to s
        \item Add all new vertices from the current frontier's edges to $V$ if they are not there already
      \end{itemize}
      \item Return $\bot$
    \end{itemize}
  \end{enumerate}
\end{itemize}

\emph{Coloring}
\begin{itemize}
  \item Inputs: Connceted graph $G = (V,E)$
  \item Outputs: a "few" coloring of the graph
  \item Possible solving algorithms:
  \begin{enumerate}
    \item Greedy: O(n + m)
    \begin{itemize}
      \item Select ordering of the colors
      \item For i in 0 to n - 1
      \begin{itemize}
        \item Coloring of vertex $v_{i}$ is the minimum color (i.e., assign each color a number) such that
        none of its edges also share this color.
      \end{itemize}
      \item return this coloring
    \end{itemize}
    \item ***BFS: I think it's $O(n + m)$, but notes aren't too clear...***
    \begin{itemize}
      \item Select ordering of vertices by performing BFS from some vertex $v_{0}$
      \item For i in 0 to n - 1
      \begin{itemize}
        \item Coloring of vertex $v_{i}$ is the minimum color (i.e., assign each color a number) such that
        none of its edges also share this color.
      \end{itemize}
      \item return this coloring
    \end{itemize}
  \end{enumerate}
\end{itemize}

\emph{Connected Components}
\begin{itemize}
  \item Inputs: Undirected graph $G = (V,E)$
  \item Outputs: a parition of V such that they are connected components
  \item Possible solving algorithms:
  \begin{enumerate}
    \item BFS: O(n + m)
    \begin{itemize}
      \item Select ordering of the colors
      \item For i in 0 to n - 1
      \begin{itemize}
        \item Coloring of vertex $v_{i}$ is the minimum color (i.e., assign each color a number) such that
        none of its edges also share this color.
      \end{itemize}
      \item return this coloring
    \end{itemize}
  \end{enumerate}
\end{itemize}

\end{document}