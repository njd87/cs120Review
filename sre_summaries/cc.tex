\documentclass{article}
\usepackage[utf8]{inputenc}
\usepackage{amsmath}

\title{CS120 Review: SRE Connected Components}
\author{NJD}
\date{2022}

\usepackage{listings}

\newenvironment{tightcenter}{%
  \setlength\topsep{0pt}
  \setlength\parskip{0pt}
  \begin{center}
}{%
  \end{center}
}

\begin{document}
\maketitle

\section*{The Algorithm}
\paragraph*{Explanation} |
If we have a graph $G=(V,E)$, we want to partition V such that each pair of vertices that are reachable from one to the other
are in the same parition (this is an iff statement). I.e., if we have a connected graph, we return only 1 V: the original V.

\paragraph*{The Idea}
If we do BFS on a random vertex $v_{0}$, we can get all connected verticies by implementing improved BFS on just the connected component
involving $v_{0}$. This is $V_{0}$. If there exists any $v_{i}$ in $V - V_{0}$, call it again on that. This should be, in total, \boxed{$O(n + m)$}

\paragraph*{The implementation}
We do this algorithm as a reduction to another algorithm called BFSLabel. First, here is how BFS-CC works:

\begin{enumerate}
  \item Create an array S, where S[$v_{i}$] = $\bot$
  \item Create a variable called $\ell$, intialize to 0
  \item For each $v_{i}$:
  \begin{enumerate}
    \item if S[$v_{i}$] == $\bot$:
    \begin{itemize}
      \item Oracle call BFSLabel to label each vertex reachable by $v_{i}$ to the value $\ell$
    \end{itemize}
    \item $\ell += 1$
  \end{enumerate}
  \item return $(\ell, S)$
\end{enumerate}

How does $BFSLabel$ work?

\begin{enumerate}
  \item Label each vertex by some input number $\ell$ iff it is reachable from starting vertex $s$; we also input a labeling array S
  \item Set $F = [s]$
  \item Pretty simple: just do a while loop for if there are vertices reachable from each vertex in $F$ SUCH THAT none of them are labeled $l$
  \begin{itemize}
    \item If yes: label $S[f_{i}] = \ell$; update $F$ to be these new verticies
    \item If no: terminate (we don't have to return since we update $S$)
  \end{itemize}
\end{enumerate}

\end{document}